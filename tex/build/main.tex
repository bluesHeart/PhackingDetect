\documentclass[11pt]{article}

\usepackage[margin=1in]{geometry}
\usepackage{setspace}
\usepackage{graphicx}
\usepackage{booktabs}
\usepackage{amsmath, amssymb}
\usepackage[round]{natbib}
\usepackage[hidelinks]{hyperref}

\graphicspath{{../figures/}}
\onehalfspacing

\title{PhackingDetect: An End-to-End LLM Workflow for Within-Paper Selective-Reporting Risk Screening}
\author{PhackingDetect contributors}
\date{\today}

\begin{document}
\maketitle

\begin{abstract}
We describe an open-source, end-to-end agent that takes a single economics/finance paper PDF and produces an evidence-grounded, referee-style selective-reporting (``p-hacking'') risk screening report. The system is designed as a practical triage tool: it does not infer intent and does not replace formal meta-scientific inference. Instead, it anchors diagnostics to specific tables/figures, pages, and searchable phrases, and it maps each flagged pattern to a clear methodological rationale from the selective-reporting literature.
\end{abstract}

\section{Introduction}

Empirical economics and finance research often involves a large ``design space'': outcome definitions, samples, control sets, functional forms, and robustness variants. Classic warnings about ad hoc specification searches emphasize that, without discipline, inference can become fragile and difficult to interpret \citep{Leamer1978, Leamer1983}. In modern applied work, these concerns interact with the widespread use of conventional significance cutoffs (e.g., $p<0.05$) and the proliferation of tests in a single paper.

The selective-reporting literature documents systematic patterns consistent with p-hacking or related forms of selection. In economics, ``bunching'' of reported results around conventional thresholds is documented in \citet{BrodeurLeSangnierZylberberg2016} and \citet{BrodeurCookHeyes2020}, and detection methods are developed in \citet{ElliottKudrinWuthrich2022}. In finance, multiple testing and recommended higher evidentiary standards are emphasized by \citet{HarveyLiuZhu2015} and \citet{Harvey2017}. At the same time, within-paper patterns can also arise from alternative mechanisms such as publication and reporting bias \citep{AndrewsKasy2019}.

\paragraph{Problem.}
Even when method papers provide clear diagnostic concepts, applying them in day-to-day research evaluation is labor intensive: a referee must scan many tables/figures, interpret robustness structure, and separate suggestive cues from definitive evidence. This creates a practical gap between methodological guidance and real-world screening.

\paragraph{Approach.}
This repository contributes an end-to-end, evidence-grounded workflow---\texttt{PhackingDetect}---that takes a single PDF and produces a referee-style \emph{within-paper} risk screening report. The goal is \emph{not} to attribute intent or misconduct; instead, the system surfaces \emph{where} risk signals appear (table/figure + page + anchors) and \emph{why} they matter under diagnostics grounded in the selective-reporting literature (Section~\ref{sec:diagnostics}).

\paragraph{Contribution.}
The project is primarily an open-source engineering artifact:
\begin{itemize}
  \item \textbf{Evidence-grounded screening:} every flagged claim is tied to a table/figure, page number, and searchable anchor phrases.
  \item \textbf{Theory mapping:} diagnostics are explicitly linked to well-cited methods papers (e.g., threshold-based evidence, multiplicity, p-curve) rather than free-form speculation.
  \item \textbf{Auditability:} the pipeline logs prompts and raw model outputs to enable independent review.
  \item \textbf{Practical cost controls:} the system is designed to triage a paper by focusing on pages rich in tables/figures and robustness content.
\end{itemize}

\paragraph{Organization.}
Section~2 reviews related methodological work. Section~3 describes the end-to-end system and how it grounds evidence in the PDF. Section~\ref{sec:diagnostics} details the diagnostic logic and its theoretical basis. Section~5 outlines an evaluation strategy for assessing evidence accuracy and screening value, and Section~6 discusses limitations and responsible use.

\section{Related Work}

PhackingDetect focuses on \emph{within-paper} screening: it examines a single paper in isolation and flags patterns that warrant scrutiny by a domain expert. This differs from many empirical studies in the p-hacking literature, which typically analyze large corpora of published papers and provide population-level stylized facts or statistical tests. Our goal is to operationalize those diagnostic concepts at the level of a single PDF and to produce an audit-ready report that points to concrete locations in the document.

\paragraph{Threshold-based evidence.}
Economics-specific evidence on ``bunching'' around conventional thresholds (e.g., $p \approx 0.05$) is documented in \citet{BrodeurLeSangnierZylberberg2016} and \citet{BrodeurCookHeyes2020}. More formal detection methods are developed in \citet{ElliottKudrinWuthrich2022}.

\paragraph{P-curve diagnostics.}
The p-curve approach (shape restrictions on significant p-values) is introduced and developed by \citet{SimonsohnNelsonSimmons2014a, SimonsohnNelsonSimmons2014b}. In our system, p-curve is treated as a downstream, heavier-weight check rather than a default claim: the agent collects the information needed to justify such analysis (which results are ``in scope'', how many tests are involved, whether the set of tests is plausibly selective), and it recommends p-curve-style follow-ups when appropriate.

\paragraph{Multiple testing and specification search.}
Multiple testing concerns and recommended higher evidentiary standards in finance appear prominently in \citet{HarveyLiuZhu2015, Harvey2017}. The broader perspective on specification searches is classic in \citet{Leamer1978, Leamer1983}.

\paragraph{Publication bias.}
Selective publication and reporting biases can mimic p-hacking-like patterns, motivating identification and correction methods such as \citet{AndrewsKasy2019}. Our agent explicitly treats publication bias as an alternative explanation when interpreting within-paper signals.

\paragraph{Operational gap addressed by this project.}
Across these literatures, a recurring challenge is that diagnostics are often presented as statistical procedures or stylized facts, while day-to-day paper evaluation is a document-centric task: the referee is confronted with a PDF that mixes narrative, tables, figures, and appendices. PhackingDetect bridges this gap by turning method concepts into a structured, evidence-grounded workflow that is (i) anchored to specific artifacts within the paper and (ii) explicit about competing explanations and recommended checks.

\section{System Overview}

The agent operationalizes ``within-paper'' screening: it evaluates a single paper in isolation, using its own tables/figures, notes, and robustness narrative. The central design goal is \emph{auditability}: the report should allow a reader to quickly verify each claim by locating the cited evidence in the PDF.

\subsection{Scope and definitions}
We use ``p-hacking'' as a shorthand for a family of selective-reporting behaviors and researcher degrees of freedom that can inflate the share of significant results. The agent does not attempt to infer intent; its output is a \emph{risk screening} document that highlights patterns consistent with common diagnostics from the literature.

\paragraph{Artifact.}
An \emph{artifact} is a table or figure (including appendix artifacts) that contains empirical results, robustness checks, event-study plots, heterogeneity panels, or other statistical summaries.

\paragraph{Anchor.}
An \emph{anchor} is a short, searchable phrase tied to the artifact (e.g., ``Table 3: Baseline results'', ``Robustness'', a column header, or a note describing star conventions). Anchors, together with page numbers, are used to ground claims.

\subsection{End-to-end workflow}
Figure~\ref{fig:pipeline} summarizes the pipeline. The workflow is deliberately modular: evidence extraction produces grounded snippets, diagnostics map those snippets to theory-grounded risk signals, and the report generator enforces a structured inference chain.

\begin{figure}[t]
\centering
\setlength{\unitlength}{1mm}
\begin{picture}(160,38)
  % boxes
  \put(0,14){\framebox(22,10){PDF}}
  \put(30,24){\framebox(38,10){Text extraction}}
  \put(30,4){\framebox(38,10){Page images}}
  \put(76,14){\framebox(40,10){Artifact discovery}}
  \put(124,14){\framebox(34,10){Diagnostics}}
  \put(124,2){\framebox(34,10){Report \& metrics}}

  % arrows
  \put(22,19){\vector(1,0){8}}
  \put(68,29){\vector(1,0){8}}
  \put(68,9){\vector(1,0){8}}
  \put(116,19){\vector(1,0){8}}
  \put(141,14){\vector(0,-1){2}}

  % labels
  \put(96,27){\makebox(0,0)[c]{tables/figures + anchors}}
  \put(141,0){\makebox(0,0)[c]{\small audit-ready provenance}}
\end{picture}
\caption{End-to-end pipeline of PhackingDetect. The report is grounded in artifact-level evidence (page + anchors) rather than free-form summaries.}
\label{fig:pipeline}
\end{figure}

\subsection{Inputs and outputs}
Input is a PDF. Outputs are:
\begin{itemize}
  \item an English diagnostic report (referee-style, risk screening);
  \item a machine-readable JSON metric payload summarizing detected signals;
  \item a full JSON bundle containing provenance (pages, anchors, intermediate steps);
  \item prompt and raw-response logs to enable auditing and reproduction.
\end{itemize}

\subsection{Evidence extraction}
The system combines lightweight PDF text extraction with rendered page images. This is important because (i) PDF text extraction can be incomplete or mis-ordered and (ii) tables/figures are often more reliably read from rendered pages than from extracted text.

\paragraph{Page triage.}
To control cost, the system prioritizes pages likely to contain tables/figures, robustness checks, and statistical evidence (e.g., pages with high numeric density, ``Table'', ``Figure'', ``Robustness'', ``Appendix'' cues). In ``expert'' mode, a language model can further refine this selection by choosing pages that appear most diagnostic given the paper's structure.

\paragraph{Grounding.}
Every flagged item must include page references and anchors. This design makes the report falsifiable: if an anchor does not locate the claimed evidence, the report is wrong and can be corrected.

\subsection{Lightweight numeric evidence extraction}
In addition to multimodal reading, the system can extract low-cost numeric evidence from the paper text and table-like structures when available. Examples include:
\begin{itemize}
  \item extracting reported $p$-values (e.g., ``$p=0.047$'') and counting values in narrow windows around conventional cutoffs (within-paper ``caliper'' summaries);
  \item reconstructing approximate test statistics and two-sided $p$-values when coefficient and standard error pairs are present (e.g., $(\hat\beta, \widehat{se}) \mapsto t=\hat\beta/\widehat{se}$);
  \item summarizing how many tests appear in a table to contextualize multiplicity risk.
\end{itemize}
These summaries are not a substitute for corpus-level inference, but they provide (i) an additional cross-check against purely narrative interpretations and (ii) a concrete bridge from within-paper evidence to threshold-based diagnostics emphasized in the literature \citep{BrodeurLeSangnierZylberberg2016, BrodeurCookHeyes2020, ElliottKudrinWuthrich2022}.

\subsection{Artifact-level screening}
The paper is decomposed into candidate artifacts. For each artifact, the agent applies a small set of diagnostics (Section~\ref{sec:diagnostics}) and records an inference chain:
\begin{center}
observation $\rightarrow$ diagnostic $\rightarrow$ why it matters (with citation) $\rightarrow$ plausible alternatives $\rightarrow$ recommended checks.
\end{center}

\paragraph{Why inference chains.}
Within-paper p-hacking signals are rarely definitive. The report therefore emphasizes logic and competing explanations: it is acceptable (and expected) for a flag to end with ``uncertain, but worth checking'', provided the evidence and rationale are explicit.

\subsection{Constrained generation and auditability}
To reduce ``hallucinated'' claims, the system uses constrained generation in multiple places: structured JSON for intermediate steps, low-temperature decoding, required fields (page/anchor), and automated validation of key outputs. All prompts and raw responses are logged, so a reviewer can inspect exactly what evidence the model saw and how it responded.

\section{Diagnostics and theoretical basis}
\label{sec:diagnostics}

The agent's report is constrained to diagnostics with a clear methodological basis. Each diagnostic is used as \emph{risk screening}, not a definitive test. The goal is to make the logic explicit and auditable: (i) what was observed in the paper, (ii) which diagnostic concept it corresponds to, and (iii) why that concept matters according to the literature.

\begin{table}[t]
\centering
\small
\begin{tabular}{p{0.22\linewidth} p{0.52\linewidth} p{0.20\linewidth}}
\toprule
\textbf{Diagnostic module} & \textbf{Within-paper cue (what the agent looks for)} & \textbf{Method basis} \\
\midrule
Near-threshold clustering & Many results that are ``just significant'' (e.g., $p\approx 0.05$ or $|t|\approx 1.96$), especially without commensurate discussion or correction. & \citet{BrodeurLeSangnierZylberberg2016, BrodeurCookHeyes2020, ElliottKudrinWuthrich2022} \\
Multiplicity risk & Many outcomes, subgroups, mechanisms, or specifications, with little discussion of multiple-testing control or higher thresholds. & \citet{HarveyLiuZhu2015, Harvey2017} \\
Specification search cues & Evidence that significance is fragile to ad hoc choices (controls, samples, functional form), with selective robustness reporting. & \citet{Leamer1978, Leamer1983} \\
Publication/reporting bias (alternative) & Patterns consistent with selection that could arise beyond within-paper p-hacking (e.g., selective reporting or publication). & \citet{AndrewsKasy2019} \\
Downstream p-curve checks & When the set of tests is plausibly selective, recommend p-curve-style follow-up rather than assert it by default. & \citet{SimonsohnNelsonSimmons2014a, SimonsohnNelsonSimmons2014b} \\
\bottomrule
\end{tabular}
\caption{Diagnostics operationalized by the agent and their methodological basis. The agent treats these as screening signals and explicitly discusses competing explanations.}
\label{tab:diagnostics_summary}
\end{table}

\subsection{Near-threshold clustering}
Within-paper concentration of test statistics near conventional cutoffs (e.g., $|t| \approx 1.96$ or $p \approx 0.05$) is a prominent empirical signature discussed in economics \citep{BrodeurLeSangnierZylberberg2016, BrodeurCookHeyes2020} and formalized by detection methods such as \citet{ElliottKudrinWuthrich2022}. Intuitively, if researchers search over specifications or outcomes, the ``best'' results are more likely to land just on the significant side of a threshold.

\begin{figure}[t]
\centering
\setlength{\unitlength}{1mm}
\begin{picture}(120,45)
  % axes
  \put(10,8){\line(1,0){100}}
  \put(10,8){\line(0,1){32}}

  % bars (conceptual)
  \put(18,8){\rule{8mm}{6mm}}
  \put(30,8){\rule{8mm}{8mm}}
  \put(42,8){\rule{8mm}{10mm}}
  \put(54,8){\rule{8mm}{12mm}}
  \put(66,8){\rule{8mm}{28mm}} % bunching just below 0.05
  \put(78,8){\rule{8mm}{9mm}}  % fewer just above 0.05
  \put(90,8){\rule{8mm}{7mm}}
  \put(102,8){\rule{8mm}{6mm}}

  % threshold marker
  \put(76,8){\line(0,1){32}}
  \put(76,42){\makebox(0,0)[c]{$p=0.05$}}
  \put(32,2){\makebox(0,0)[c]{smaller $p$}}
  \put(92,2){\makebox(0,0)[c]{larger $p$}}
\end{picture}
\caption{Conceptual illustration of near-threshold clustering: disproportionately many ``just significant'' results just below $p=0.05$ relative to just above. The agent looks for within-paper analogs of this pattern in reported tables/figures.}
\label{fig:threshold_bunching}
\end{figure}

\paragraph{Operationalization in the agent.}
The report flags an artifact when it observes multiple ``borderline'' results (e.g., stars appearing exactly at conventional cutoffs, or test statistics close to the threshold) and when the paper's narrative does not provide a clear reason to expect such a distribution (e.g., pre-specified primary outcomes with appropriate adjustment). When available, the agent supplements visual cues with lightweight within-paper numeric summaries (e.g., counts of extracted $p$-values in narrow bins around $0.05$), which are conceptually aligned with threshold-based diagnostics in \citet{BrodeurLeSangnierZylberberg2016, BrodeurCookHeyes2020}.

\paragraph{Competing explanations and checks.}
Threshold clustering can arise from rounding, discrete test statistics, standardized reporting conventions, or genuine effects that are small and precisely estimated. The report therefore recommends follow-ups rather than conclusions, such as: checking the full set of outcomes tested, applying multiple-testing correction where appropriate, and verifying whether the set of reported tests corresponds to a pre-specified analysis plan.

\subsection{Multiplicity and garden-of-forking-paths risk}
When a paper reports many outcomes, subgroups, and alternative specifications, the chance of at least one false positive rises without multiple-testing adjustments. This is emphasized in finance and empirical practice discussions \citep{HarveyLiuZhu2015, Harvey2017}. In within-paper screening, multiplicity risk matters because it changes how ``a few significant findings'' should be interpreted.

\paragraph{Operationalization in the agent.}
The agent scans for cues of large testing burdens: many columns/outcomes, many heterogeneity splits, many mechanisms, large robustness matrices, and extensive appendix tables. It then checks whether the paper documents any correction (FWER/FDR), a pre-specified primary outcome strategy, or stricter thresholds justified by the testing environment.

\paragraph{Interpretation.}
The report treats missing correction as a \emph{risk factor}, not a defect. In many settings, authors may reasonably present exploratory evidence; the key issue is whether the interpretation and claims match the inferential environment.

\subsection{Specification search cues}
Classic concerns about ad hoc model search and fragile inference are articulated in \citet{Leamer1978, Leamer1983}. A within-paper signal of specification searching is that key claims hinge on a narrow set of choices, while alternative plausible choices are unreported or relegated to appendices without discussion.

\paragraph{Operationalization in the agent.}
The report flags ``fragility'' when it observes: (i) large swings in significance across nearby specifications; (ii) cherry-picked robustness variants (only significant variants highlighted); or (iii) narrative emphasis inconsistent with the overall robustness pattern.

\paragraph{Checks.}
Recommended checks include transparent disclosure of the considered specification set, robustness summaries that include null results, and (where feasible) preregistered or externally validated specifications.

\subsection{Alternative explanations: publication and reporting bias}
Patterns that resemble p-hacking can also arise from selective publication/reporting. \citet{AndrewsKasy2019} provides a modern framework for identification and correction. For a single paper, this means that ``too many significant results'' is ambiguous: it could reflect within-paper flexibility, selection into the published record, or both.

\paragraph{Operationalization in the agent.}
The report explicitly states when alternative explanations are plausible and avoids intent attribution. It recommends framing claims in a way consistent with uncertainty and, when possible, triangulating with external evidence (e.g., replication, pre-analysis plans, or out-of-sample validation).

\subsection{Downstream p-curve checks}
P-curve methods \citep{SimonsohnNelsonSimmons2014a, SimonsohnNelsonSimmons2014b} provide shape-based diagnostics of evidentiary value. In practice, their interpretation depends on defining the relevant set of tests and understanding how selective that set may be.

\paragraph{Role in this system.}
Rather than assert a p-curve conclusion by default, the agent treats p-curve as a follow-up that becomes more appropriate when it has collected enough information to define a plausible set of primary tests (e.g., main-table coefficients corresponding to a pre-specified claim). The agent's report therefore records: which tests appear central, which are robustness or exploratory, and how multiplicity might shape p-curve interpretation.

\section{Evaluation plan (draft)}

Because the system is intended as an open-source screening tool, evaluation should emphasize \emph{auditability} and \emph{usefulness} rather than a single headline accuracy number. A credible evaluation plan should answer four practical questions: (i) does the report correctly point to the cited evidence, (ii) are the diagnostic rationales grounded in the literature, (iii) does the tool help an expert triage faster, and (iv) what does it cost to run?

\subsection{Evidence-grounding accuracy}
The primary failure mode for document-level agents is ungrounded claims. We therefore treat evidence correctness as the first-class metric:
\begin{itemize}
  \item \textbf{Anchor validity:} does the quoted anchor phrase appear on the claimed page and near the claimed table/figure?
  \item \textbf{Artifact match:} does the cited table/figure label correspond to the described content?
  \item \textbf{Numeric spot checks:} when numeric claims are made (e.g., ``borderline $p$''), are they consistent with the artifact?
\end{itemize}
These checks can be audited by a reviewer without re-running the entire paper's analysis.

\subsection{Diagnostic usefulness (expert agreement)}
Even perfectly grounded reports can be unhelpful if they flag irrelevant issues. A natural evaluation is to sample papers and ask domain experts to rate:
\begin{itemize}
  \item whether each flagged item is \emph{worth checking};
  \item whether the proposed follow-up checks are appropriate;
  \item whether the report correctly distinguishes ``suggestive'' from ``definitive'' evidence.
\end{itemize}
We emphasize that disagreement is expected: the tool is a triage assistant, and papers vary widely in norms, settings, and identification strategies.

\subsection{Ablations and robustness}
To build confidence that the system is not relying on a single brittle heuristic, an evaluation should include ablations:
\begin{itemize}
  \item \textbf{Text-only vs. multimodal:} compare evidence grounding when page images are removed.
  \item \textbf{Page triage sensitivity:} vary the maximum number of analyzed pages and measure changes in detected artifacts.
  \item \textbf{Diagnostic coverage:} remove one diagnostic module at a time (Table~\ref{tab:diagnostics_summary}) and assess what signals disappear.
\end{itemize}

\subsection{Cost profile}
Cost depends on paper length, number of artifacts, and how much context is provided per artifact. A transparent evaluation should report:
\begin{itemize}
  \item wall-clock time and number of model calls per paper,
  \item token usage (prompt/completion) and variance across papers,
  \item how cost scales with ``expert'' settings (pages and artifacts analyzed).
\end{itemize}

\begin{table}[t]
\centering
\small
\begin{tabular}{p{0.28\linewidth} p{0.62\linewidth}}
\toprule
\textbf{Evaluation target} & \textbf{How to measure (auditable)} \\
\midrule
Evidence correctness & Manual audit of page + anchor; spot-check a small random sample of flags. \\
Screening value & Expert ratings of ``worth checking'' and ``good follow-up suggestion''. \\
Robustness & Ablations over page budget and modality; measure stability of key flags. \\
Cost & Tokens and wall-clock time per paper under standardized settings. \\
\bottomrule
\end{tabular}
\caption{A practical evaluation checklist for an audit-ready within-paper screening agent.}
\label{tab:evaluation_checklist}
\end{table}

\subsection{Reproducibility}
For an agent that depends on large language models, reproducibility requires engineering choices: fixed decoding parameters where possible, deterministic caching of intermediate steps, and complete logging of prompts and raw responses. The repository is designed to make each report auditable and to allow re-running with the same model endpoint to assess stability.

\section{Limitations}

First, a within-paper screening tool cannot infer intent and should not be used to accuse misconduct. It can only surface patterns that are \emph{consistent with} selective reporting, specification search, or related forms of selection. Users should treat the output as triage for expert review rather than adjudication.

Second, document understanding is imperfect. PDF parsing and OCR-like reading of rendered pages can fail on scanned documents, low-resolution tables, unusual fonts, multi-column layouts, and appendices. A key design choice in this repository is therefore to include page numbers and anchors so that users can quickly verify (or falsify) each claim.

Third, many diagnostics are suggestive rather than definitive. Near-threshold clustering, multiplicity, and robustness structure are informative cues, but they do not mechanically imply p-hacking. The appropriate response is typically \emph{additional checks} (stronger thresholds, preregistration, multiple-testing correction, out-of-sample validation, or more transparent robustness reporting), not a binary judgement.

Fourth, within-paper screening has a fundamental identification problem: it cannot separate within-paper selection from selection into the published record. Patterns consistent with p-hacking may also be consistent with publication bias or reporting conventions, which motivates careful alternative-explanation language \citep{AndrewsKasy2019}.

\paragraph{Responsible use.}
The tool is intended to help reviewers and researchers prioritize scrutiny, not to generate accusations. Reports should be phrased in probabilistic, non-accusatory terms (``risk signal'', ``worth checking'') and should always include the cited evidence and recommended follow-up checks.

\section{Conclusion}

PhackingDetect is an open-source, end-to-end workflow for producing evidence-grounded selective-reporting risk screening reports from a single paper PDF. It is designed to be practical (low-friction to run) while remaining anchored in well-known diagnostics from the p-hacking and selective-reporting literature. The core contribution is not a new statistical test, but an engineering bridge: transforming method-paper concepts into a structured, auditable document-level workflow that points to concrete evidence in tables and figures.

Future work includes more systematic evaluation (Section~5), stronger artifact extraction for complex PDFs, and additional downstream statistical modules (e.g., more formal implementations of p-curve-style inference) when the relevant set of tests can be defined transparently and responsibly.


\bibliographystyle{apalike}
\bibliography{../references}

\end{document}
